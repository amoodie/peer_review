\documentclass{article}


\usepackage{peer_review}
\title{A review of: \\ Manuscript title that I am reviewing goes here}
% \title{Co-author review of: \\ Manuscript title of my co-author's paper goes here}
\author{Review by: Author O. Thereview}



\begin{document}
\maketitle


\section{Overall comments}
This is just a regular section. 
We could use this to put overall comments about the manuscript that we want the author to read before getting into the details.

\begin{sectioncomments}[A Section Title]
    \comm{This is a comment using \texttt{$\backslash$comm} that has a line number given as an optional parameter after the main comment.
    Comments are automatically numbered at the left.}[10]
    \comm{This is a comment using \texttt{$\backslash$comm} without a given line number and come more blind text. 
    Far far away, behind the word mountains, far from the countries Vokalia and Consonantia, there live the blind texts. 
    Separated they live in Bookmarksgrove right at the coast of the Semantics, a large language ocean.}
    \comm{This is also a comment using \texttt{$\backslash$comm}, but below is some generic sub-comment where you want to add an additional note. Maybe it's another suggestion that comments on the same sentence, but about a different idea.}[71]
        \subcomm{This is a \texttt{$\backslash$subcomm}. 
        Note that subcomm does not change the font size, and also supports an optional line number field. This environment could be as long as needed.}[88]
\end{sectioncomments}

\begin{sectioncomments}[Another Section Title]
    \comm{Sometimes you just want to suggest a rewrite of a small section of the manuscript instead of line by line edits you can use the \texttt{$\backslash$rewrite} command. 
    Note that \texttt{$\backslash$rewrite} will only work inside of \texttt{$\backslash$comm}.
    For example, say: I have rewritten the last two paragraphs as follows:}[92]
        \rewrite{\forceindent I am now inside a \texttt{$\backslash$rewrite}. Notice that this shrinks the font size! 
        Far far away, behind the word mountains, far from the countries Vokalia and Consonantia, there live the blind texts. 
        Separated they live in Bookmarksgrove right at the coast of the Semantics, a large language ocean. 
        A small river named Duden flows by their place and supplies it with the necessary regelialia. 
        It is a paradisematic country, in which roasted parts of sentences fly into your mouth. 
        Even the all-powerful Pointing has no control about the blind texts it is an almost unorthographic life.
        One day however a small line of blind text by the name of Lorem Ipsum decided to leave for the far World of Grammar. 
        \newpara The Big Oxmox advised her not to do so, because there were thousands of bad Commas, wild Question Marks and devious Semikoli, but the Little Blind Text didn’t listen. 
        She packed her seven versalia, put her initial into the belt and made herself on the way. 
        When she reached the first hills of the Italic Mountains, she had a last view back on the skyline of her hometown Bookmarksgrove, the headline of Alphabet Village and the subline of her own road, the Line Lane.}
    \comm{note that in the above \texttt{$\backslash$rewrite} example, you must use \texttt{$\backslash$forceindent} and \texttt{$\backslash$newpara} to get an indent at the first line and break to a new paragraph with indent, respectively. 
    \texttt{$\backslash$rewrite} also supports an optional line number at the end of the command.}
\end{sectioncomments}


\begin{sectioncomments}[Whoa Another Section Title]
    \comm{We can use quotes to show what exactly we are commenting on, for example: ``This sentence poorly written it is.'' should be ``This sentence is poorly written''.}
    \comm{You can use equations in the text of a \texttt{$\backslash$comm}, just like you would in any other environment: 
    \begin{equation}\label{RC1}
        E = mc^2
    \end{equation}
    Note that equations will be numbered and prefixed with an ``R'' denoting a \textbf{r}eview equation, distinct from the equations in the document you are reviewing.}
\end{sectioncomments}


\begin{sectioncomments}*[This will be printed exactly as written]
    \comm{In the above section head, we specified an asterisk before the optional field. This causes the optional text given to be printed exactly as written instead of prefixed with ``Comments about'' and put in quotation marks.}[234]
    \comm{Far far away, behind the word mountains, far from the countries Vokalia and Consonantia, there live the blind texts. 
    Separated they live in Bookmarksgrove right at the coast of the Semantics, a large language ocean. 
    A small river named Duden flows by their place and supplies it with the necessary regelialia. 
    It is a paradisematic country, in which roasted parts of sentences fly into your mouth.}[494]
\end{sectioncomments}

\begin{sectioncomments}
    \comm{In the above section head, no arguments were provided, so we get a default printing of ``Comments''. 
    Note that specifying the asterisk with no optional argument also produces the default ``Comments''.}[365]
    \comm{Far far away, behind the word mountains, far from the countries Vokalia and Consonantia, there live the blind texts. 
    Separated they live in Bookmarksgrove right at the coast of the Semantics, a large language ocean. 
    A small river named Duden flows by their place and supplies it with the necessary regelialia. 
    It is a paradisematic country, in which roasted parts of sentences fly into your mouth.}[543]
\end{sectioncomments}


\end{document}